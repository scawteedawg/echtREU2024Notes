\section{Day 2}
The plan for the day is as follows. 
\begin{enumerate}
    \item Spectra
    \item Algebra
\end{enumerate}
The plan for tomorrow is to do the following. 
\begin{enumerate}
    \item An example of a spectral sequence. This will use some group cohomology.
    \item Showing the \texttt{spectralsequences} package in \LaTeX, and maybe some \LaTeX practice.
\end{enumerate}
We aren't trying to give homework! This is your job! There is no need to work outside of working hours. Please try to get \LaTeX installed into your computer by tomorrow so we can practice it! Feel free to ask Irina, Jack, and Scotty for help. 

\lineyspace
\subsection{Lecture 1 (Irina Bobkova): Spectra}

\begin{enumerate}
    \item $\Sigma$, the reduced suspension, is a functor among topological spaces. $\pi_\ast$ is another functor from topological spaces to groups. There is a map, \begin{center}
      \begin{tikzcd}[ampersand replacement = \&]
        {\pi_{r+n}(S^n)}\ar[r,"\Sigma"]\& {\pi_{r+n+1}(S^{n+1})} \\ [-20pt]
        {\parenths{S^{r+n}\xr{f}S^n}}\ar[r,mapsto] \& {\parenths{\underbrace{\Sigma S^{r+n}}_{S^{r+n+1}}\xr{\Sigma f}\underbrace{\Sigma S^n}_{S^{n+1}}}}\\[-20pt]
        f\ar[r,mapsto]\& \Sigma f
      \end{tikzcd}
    \end{center}
    $\pi_{n+r}(S^n)$ only depends on $r$ for $n$ large enough because of the {Freudenthal Suspension theorem}. Here, we get the definition of the stable homotopy groups of spheres, which we see as $$\pi_r^{st}:=\varinjlim_{n\geq 0}\parenths{\cdots\xr{\Sigma}\pi_{n+r}(S^n)\xr{\Sigma}\pi_{n+1+r}(S^{n+1})\xr{\Sigma}\cdots}$$
    More generally, for any space $X$, you get $$\pi_{n+r}(\Sigma^n X)\xr{\cong}\pi_{n+1+r}(\Sigma^{n+1}X)$$ as long as $n$ is large enough. 
    \begin{theorem}{Freudenthal Suspension Theorem}{}\index{Freudenthal Suspension Theorem}
        Let $X$ be $(n-1)$ connected ($\pi_k(X)=0$ for $k = 0,\dots, n-1$). Note that we have maps $$\Sigma:\pi_qX\to \pi_{q+1}\Sigma X.$$ Then, when $q<2n-1$, $\Sigma$ is a bijection and when $q = 2n-1$, it is a surjection.
    \end{theorem}
    For us today, a ``space'' means a compactly generated, weakly Hausdorff topological space. This excludes spaces that are ``bad.'' \href{https://en.wikipedia.org/wiki/Compactly_generated_space}{See here for more about compactly generated} spaces and  \href{https://en.wikipedia.org/wiki/Weak_Hausdorff_space}{see here for weakly Hausdorff} spaces. 
    \item A spectrum $E$ is 
    \begin{itemize}
        \item A collection of spaces $\sets{E_n}_{n=0}^\infty$
        \item Structure maps $\alpha_n:\Sigma E_n\to E_{n+1}$. 
    \end{itemize}
    \begin{example}{}{}
        \begin{multicols*}{2}
            \begin{itemize}
                \item Given a space $X$, we can cook a suspension spectrum $\Sigma^\infty X$, with 
                \begin{itemize}
                    \item Spaces: $\sets{(\Sigma^\infty X)_n}_n$ where $(\Sigma^\infty X)_n:=\Sigma^n X$
                    \item Structure maps: $\alpha_n:\Sigma(\Sigma^n X)\xr{\cong}\Sigma^{n+1}X$. 
                \end{itemize}
                \columnbreak
                \item $H_\ast(X;\mZ)$ comes from a spectrum called an {Eilenberg-MacLane spectrum}. The spectrum is denoted $H\mZ$ and the data of $H\mZ$ is given by 
                \begin{itemize}
                    \item Spaces: $H\mZ_n:=K(\mZ,n)$ the Eilenberg-MacLane space for $\mZ$ in degree $n$. This means that if a space $A$ is a $K(\mZ,n)$, then $\pi_n(A) = \mZ$ and $\pi_k(A)=0$ for all $k\neq n$.\footnote{$\mZ$ is a $K(\mZ,0)$, $S^1$ is a $K(\mZ,1)$, and $\mCP^\infty$ is a $K(\mZ,2)$. $S^0$ is a $K(\mZ/2,0)$, and $\mRP^\infty$ is a $K(\mZ/2,1)$.} 
                    \item Structure maps: $\Sigma K(\mZ,n)\to K(\mZ,n+1)$. 
                \end{itemize}
            \end{itemize}
        \end{multicols*}
    \end{example}
    \begin{definition}{Eilenberg MacLane spaces}{}
        For any Abelian group $G$ and $n\geq 0$ there exists a space $X$ which is considered $K(G,n)$, or an \defindex{Eilenberg MacLane space}, such that $$\pi_k(X) =\begin{cases}
            G & k = n\\
            0 & \text{else}.
        \end{cases}$$
        See the \href{https://math.mit.edu/~mbehrens/18.906/notes/lec15.pdf}{construction here.}
    \end{definition}
    \item Homotopy Groups of a spectrum $E$
    $$\pi_{n+r}E_n\xr{\Sigma}\pi_{n+1+r}E_{n+1}\xr{\alpha_n}\pi_{n+1+r}E_{n+1}.$$ On the left, the codimension (difference in the index) is $r$, and that matches with the group on the right! The codimension is also $r$ there. The $r$th stable group should hopefully convince you at the very least that there is something to the idea that if we did this forever, the homotopy group would ``stabilize'' once $n$ gets big enough!

    \begin{definition}{\defindex{Homotopy groups of a spectrum}}{}
        Let $E$ be a spectrum. The $r$th homotopy group of $E$ is defined to be:
        $$\pi_r E:=\underset{n\to\infty}{\colim}~\pi_{n+r}E_n.$$
    \end{definition}
    If we take $E = \Sigma^\infty X$, then $\pi_n E = \pi_n^{st} X$, the stable homotopy groups of $X$. This definition recovers the definition of stable homotopy groups! Great!
    \item Maps between spectra. 
    
    \begin{definition}{Maps of spectra, version 1}{}
        Let $E, F$ be two spectra. A \defindex{map between spectra} $f:E\to F$ is a collection of maps $f_n:E_n\to F_n$ such that these $f_n$ are compatible with the structure maps ($\eps_n$ for $E$ and $\phi_n$ for $F$), i.e. for each $n$: 
        \begin{center}
          \begin{tikzcd}[ampersand replacement = \&]
            \Sigma E_{n}\ar[r,"\eps_n"]\ar[d,"\Sigma f_n"']\ar[dr,phantom,"\circlearrowleft"]\& E_{n+1}\ar[d,"f_{n+1}"]\\
            \Sigma F_n \ar[r,"\phi_n"] \& F_{n+1}
          \end{tikzcd}
        \end{center}
    \end{definition}
    This definition seems good! Let's try another definition and then we can compare and contrast which will be better for our purposes. Which definition will have better theorems?

    Here is another construction. 
    \begin{definition}{Maps between spectra, version 2}{}
        What if instead, we take maps $f_n:E_n\to F_{n-r}$ where we decrease degree by $r$. We'd still like this to be compatible with structure maps. 
        \begin{center}
            \begin{tikzcd}[ampersand replacement = \&]
              \Sigma^r E_{n-r}\ar[r,"\eps"]\ar[d,"\Sigma f_{n-r}"']\ar[dr,phantom,"\circlearrowleft"]\& E_{n}\ar[d,"f_{n}"]\\
              \Sigma^r F_{n-2r} \ar[r,"\phi"] \& F_{n-r}
            \end{tikzcd}
          \end{center}
    \end{definition}
    Let's explore this a little bit. 

    Consider the map $f:\mS\to \mS$ which is a degree 2 map, so we have maps $S^{n+2}\to S^n$. Note that $S^2\to S^0$ is nullhomotopic. $S^3\to S^1$ is nullhomotopic. However, $S^4\to S^2$ is not nullhomotopic and can be represented by $\eta^2$ (look up the Hopf map! There are several cool links if you ask one of us about them). 

    Here is yet another construction!
    \begin{definition}{Maps between spectra, version 3}{}
        A map of spectra $f:E\to F$ of degree $r$ is a homotopy class of functions of spectra $f:E\to F$ of degree $r$ where the function is defined ``in the limit.'' Find the maps between high enough $E_N$, and worry about the early ones later on. Scotty heard from his advisor that this philosophy is ``cells now, maps later.''
    \end{definition}
    \alerttext{NOTATION:} When we talk about maps from spectra to spectra of degree $r$, we denote the collection of homotopy classes of maps of degree $r$ between the two spectra as $$[E,F]_r.$$
    \item Homology and cohomology. Let $E,X$ be spectra. 
    
    \begin{definition}{Homology and Cohomology}{}
        The $E$-cohomology of $X$ in degree $r$ is $$E^r(X):=[X,E]_{-r},$$ maps of spectra $X\to E$ which lower degree by $r$. 
    
        The $E$-homology of $X$ in degree $r$ is $$E_r(X)=[\mS,E\wedge X]_r = \pi_r(E\wedge X).$$
    \end{definition}
    When I say coefficients of a spectrum $E$, what I really mean is $E_r(\ast) = \pi_r E = E^{-r}(\ast)$. This is usually written $E_\ast$ to collect all $r$ into one neat little package. By this we mean $$E_\ast = \bigoplus_{r\in \mZ} E_r(\ast).$$
    \item Given a map of spectra $f:X\to Y$, define $$(Y\cup_f CX)_n:=Y_n\cup_{f_n}(I_+\wedge X_n).$$ This gives us a long cofiber sequence of spectra $$\cdots\to\Omega X\to \Omega Y\to\Omega \text{Cof } fX\xr{f}Y\xr{i}\underbrace{Y\cup_f CX}_{\text{Cof }f}\to \underbrace{(Y\cup_f CX)\cup_i CY}_{\Sigma X}\to \Sigma Y\to \cdots.$$ Here $\Omega(-) = \mathsf{Map}(\mS^1,-)$, the loops. And in spectra, $\Omega$ is something like $\Sigma\inv$. 
\end{enumerate}
Here, we have been using a lot from the blue book: \cite[pg. 123, \S 3]{adams1974stable}. Here is a \href{https://people.math.rochester.edu/faculty/doug/otherpapers/Adams-SHGH.pdf}{pdf link}. 

\newpage

\subsection{Lecture 2 (Jack Carlisle): Homological Algebra and Group Cohomology}

\subsubsection{Homological Algebra}
\begin{definition}{Chain complex}{}
    A \defindex{chain complex} is $$\cdots \xr{d_{n+2}}C_{n+1}C_{n+1}\xr{d_{n+1}}C_{n}\xr{d_n}C_{n-1}\xr{d_{n-1}}\cdots$$ such that 
    \begin{itemize}
        \item[-] For all $i$, $d_{i-1}d_i = 0$. Another way people write this is $d\circ d =0$ or $d^2 =0$.  
    \end{itemize}
\end{definition}
\begin{example}{}{}\label{example:w1d2ex1}
    $$0\to \mZ\xr{\by 5} \mZ\xr{\by 0} \mZ\xr{\by 2}\mZ\to 0.$$
    Notice that composing any two differentials, we get 0! This means that we have a chain complex. 
\end{example}
\begin{example}{}{}\label{example:w1d2ex2}
    An exact sequence of $R$-modules $$\cdots \xr{f_{n+2}} M_{n+1}\xr{f_{n+1}} M_n\xr{f_{n}} M_{n-1}\xr{f_{n-1}} M_{n-2}\xr{f_{n-2}} \cdots$$ is an example of an exact sequence. By exactness, we get that $$f_{n+1}(m) \in \Im(f_{n+1}) = \ker(f_n)\implies f_nf_{n+1}m = 0$$ so the composition of any two is zero. 

    Here recall that exactness means that for all $n$, $$\ker f_n = \Im f_{n+1}.$$
\end{example}

\begin{definition}{Homology of a chain complex}{}
    The \defindex{homology of a chain complex} $C = (C_\bullet, d_\bullet)$ is defined to be $$H_n(C):= \frac{\ker d_n}{\Im d_{n+1}}, \qquad H_\ast(C):=\bigoplus_{n\in \mZ}H_n(C).$$
\end{definition}

\begin{example}{Example above continued}{}
    Let's calculate the homology from \ref{example:w1d2ex1}, $$\underbrace{0}_4\to \underbrace{\mZ}_3\xr{\by 5} \underbrace{\mZ}_2\xr{\by 0} \underbrace{\mZ}_1\xr{\by 2}\underbrace{\mZ}_0\to \underbrace{0}_{-1}.$$ From this, we get 
    \begin{align*}
        H_0(C)& =\frac{\ker (\mZ\to 0)}{\Im(\mZ\xr{\by 2}\mZ)} = \mZ/2\\
        H_1(C)& =\frac{\ker (\mZ\xr{\by 2} \mZ)}{\Im(\mZ\xr{\by 0}\mZ)} = 0\\
        H_2(C)& =\frac{\ker (\mZ\xr{\by 0} \mZ)}{\Im(\mZ\xr{\by 5}\mZ)} = \mZ/5\\
        H_3(C)& =\frac{\ker (\mZ\xr{\by 5} \mZ)}{\Im(0\to \mZ)} = 0\\
    \end{align*}
    This means that $$H_n(C):= \begin{cases}
        \mZ/2 & n = 0\\
        \mZ/5 & n= 2\\
        0 & \text{otherwise}
    \end{cases}.$$
\end{example}
\begin{definition}{Map of Chain complexes}{}
    A \defindex{map of chain complexes} $$f:C\to D$$ is
    \begin{itemize}
        \item A collection of maps $f_i:C_i\to D_i$
    \end{itemize}
    such that 
    \begin{itemize}
        \item[-]   \begin{tikzcd}[ampersand replacement = \&]
            C_i\ar[r,"d_i^C"]\ar[d,"f_i"]\& C_{i-1}\ar[d,"f_{i-1}"]\\
            D_i\ar[r,"d_i^D"]\& D_{i-1}
          \end{tikzcd}, or in equation form, we have $d_i^D \circ f_i = f_{i-1}\circ d_i^C$. 
    \end{itemize}
\end{definition}
\begin{lemma}{}{}
    If $f:C\to D$ is a chain map, then $f$ induces a map $$H_\ast(f):H_\ast(C)\to H_\ast(D).$$
\end{lemma}
\begin{proof}The proof should come from this diagram! Take a look at how to make  it work. 
    \begin{center}
      \begin{tikzcd}[ampersand replacement = \&]
        \&C_n\ar[r,"f_n"]\& D_n\&\\[-10pt]
        \&\ker(d_n^C)\ar[u,phantom, "\cup"]\ar[r]\ar[dr]\&\ker(d_n^D)\ar[u,phantom,"\cup"]\ar[d,twoheadrightarrow]\\
        H_n(C) \ar[r,phantom,"="]\&[-20pt] \ds\frac{\ker d_n^c}{\Im d_{n+1}^c}\ar[r]\& \ds \frac{\ker d_n^D}{\Im d_{n+1}^D}\ar[r,phantom,"="]\&[-20pt] H_n(D)
      \end{tikzcd}
    \end{center}
    Our condition implies that $f(\ker(d_n^C))\subset \ker d_n^D$. Also it means that $f(\Im(d_{n+1}^C))\subset \Im(d_{n+1}^D)$. This completes our proof, so long as you work through the details!
\end{proof}

\begin{definition}{Chain homotopy}{}
    Suppose $f,g:C\to D$ are chain maps. A \defindex{chain homotopy} $h$ from $f$ to $g$ is a collection of maps $h_i:C_i\to D_{i+1}$ such that $$f_i - g_i = d h_i + h_{i-1} d.$$
\end{definition}

\begin{fact}{}{}
    If $f,g:C\to D$ are homotopic chain maps, then $$H_\ast(f) = H_\ast(g)$$ as maps between $H_\ast(C)\to H_\ast(D)$. 

    \alerttext{This will appear in your problem set!}
\end{fact}

Hey, Jack! What about cohomology?

\begin{remark}{}{}
    Sometimes we will work instead of with chain complexes, we'll work with \defindex{cochain complexes}. These are the same as chain complexes except the differentials \emph{INCREASE} degree. Let's compare
    \begin{align*}
        \text{Chain Complex }& \cdots \to C_{n+1}\xr{d_{n+1}}C_n\xr{d_n}C_{n-1}\xr{d_{n-1}}\cdots\\
        \text{Cochain Complex }& \cdots \to C^{n-1}\xr{d^{n-1}}C^n\xr{d^n}C^{n+1}\xr{d^{n+1}}\cdots\\
    \end{align*}
    If $C = (C^\bullet, d^\bullet)$ is a cochain complex, then the cohomology of $C$ is $$H^n(C): = \frac{\ker d^n}{\Im d^{n-1}}.$$
\end{remark}

\newpage
\subsubsection{Group Cohomology}

Group cohomology is a ``machine'' which eats a finite Abelian group $G$ an spits out a cohomology group $H^\ast(G;\mZ)$. More generally, we can start with a finite group $G$ and a coefficient group $M$ to get the group cohomology $G$ with coefficients in $M$ ($M$ ends up being a $\mZ[G]$-module), written as $H^\ast(G;M)$. 

But Jack! How do we calculate this?

\begin{definition}{Calculation of Group Cohomology}{}
    \textbf{Step 1:} Find a free resolution of $\mZ$ by free $\mZ[G]$-modules. In other words, we'd like an exact sequence
    $$\cdots\to F_2\to F_1\to F_0\to \mZ\to 0$$
    with $F_i$ a free $\mZ[G]$-module. In other words, we want $F_i\cong \bigoplus_{?}\mZ[G]$.\footnote{Recall that $\mZ[G] = \sets{\sum_1^n a_i g_i\mid a_i\in \mZ, g_i\in G}$. Another way to say this is all finite sums of elements of $G$ where multiplication works the way you'd want it. By this, I specifically mean $(5g+2g')\times (6h) = 30(gh)+12(g'h)$. \\For a specific example, $\mZ[C_2] = \mZ[x]/(x^2-1 = 0)$. \\For the general definition, $\mZ[G] = \mZ[x_g\mid g\in G]/ x_gx_h = x_{gh}, x_e = 1$. \\Note that $\mZ$ is a $\mZ[G]$-module by $G$ acting trivially on $\mZ$.} 

    \medskip
    \textbf{Step 2:} Apply the functor $\Hom_{\mZ[G]}(-,M)$. Now, take the \emph{just the free part} of our resolution 
    \begin{center}
      \begin{tikzcd}[ampersand replacement = \&]
        {\cdots\to F_2\to F_1\to F_0}\ar[d,"{\Hom_{\mZ[G]}(-,M)}"]\\
        {\cdots \from \Hom_{\mZ[G]}(F_2,M)\from \Hom_{\mZ[G]}(F_1,M)\from \Hom_{\mZ[G]}(F_0,M).}
      \end{tikzcd}
    \end{center}
    This second sequence is a cochain complex of Abelian groups ($\mZ[G]$-modules).
    
    \medskip
    \textbf{Step 3:} Take cohomology groups of this cochain complex!\footnote{This is also recognizable as the \href{https://en.wikipedia.org/wiki/Ext_functor}{Ext functor}!} $$H^n(G;M): = \text{The $n$th cohomology group of the cochain complex above.}$$
\end{definition}
\begin{example}{Example calculation of group cohomology}{}
    Let $G = C_2$ and suppose $M =\mF_2$ with trivial $C_2$-action ($C_2$ does nothing). Let's calculate $H^\ast(C_2;\mF_2)$. 

    \medskip
    \textbf{Step 1:} Let's find our free resolution. Lucky for us, Jack found one already, and let's see what it is and verify this is in fact a good resolution. 

    For notation, let $\mZ[C_2] = \mZ[y]/(y^2-1)$. The following sequence is a free resolution: 
    $$\xr{(-)\cdot(y-1)}\underbrace{\mZ[C_2]}_{F_2}\xr{(-)\cdot(y+1)}\underbrace{\mZ[C_2]}_{F_1}\xr{(-)\cdot (y-1)}\underbrace{\mZ[C_2]}_{F_0}\xr{y\mapsto 1}\mZ\to 0.$$ Prove that this is an exact sequence!

    \medskip
    \textbf{Step 2:} Apply $\Hom_{\mZ[C_2]}(-,\mF_2)$. We'll drop the subscript to declutter. 

    $$\cdots \from \Hom(\mZ[C_2],\mF_2)\from \Hom(\mZ[C_2],\mF_2)\from \Hom(\mZ[C_2],\mF_2)$$ which corresponds to $$\cdots\xleftarrow{(-)\cdot 0} \mF_2\xleftarrow{(-)\cdot 0} \mF_2\xleftarrow{(-)\cdot 0} \mF_2\xleftarrow{(-)\cdot 0} \mF_2\xleftarrow{(-)\cdot 0} \mF_2\xleftarrow{(-)\cdot 0} \mF_2.$$
    
    \medskip
    \textbf{Step 3:} Let's take cohomology groups! 
    \begin{align*}
        H^0(C_2;\mF_2)& = \frac{\ker (\mF_2\xr{0}\mF_2)}{\Im(0\to \mF_2)} = \mF_2\\
        H^1(C_2;\mF_2)& = \frac{\ker (\mF_2\xr{0}\mF_2)}{\Im(\mF_2\xr{0} \mF_2)} = \mF_2\\
        H^2(C_2;\mF_2)& = \frac{\ker (\mF_2\xr{0}\mF_2)}{\Im(\mF_2\xr{0} \mF_2)} = \mF_2\\
    \end{align*}
    Finally, this gives us $$H^n(C_2;\mF_2) = \begin{cases}
        \mF_2 & n\geq 0\\ 
        0 & n< 0.
    \end{cases}$$
\end{example}

\begin{example}{$H^\ast (C_2;\mZ)$ where $\mZ$ has a trivial $C_2$ action ($C_2$ does nothing)}{}
    \textbf{Step 1:} Take this free resolution: $$\xr{(-)\cdot(y-1)}\underbrace{\mZ[C_2]}_{F_2}\xr{(-)\cdot(y+1)}\underbrace{\mZ[C_2]}_{F_1}\xr{(-)\cdot (y-1)}\underbrace{\mZ[C_2]}_{F_0}\xr{y\mapsto 1}\mZ\to 0.$$

    \medskip 
    \textbf{Step 2:} Apply $\Hom_{\mZ[C_2]}(-,\mZ)$ to the complex to get a cochain complex $$\cdots \xleftarrow{(-)\cdot 2} \mZ\xleftarrow{(-)\cdot 0} \mZ\xleftarrow{(-)\cdot 2} \mZ\xleftarrow{(-)\cdot 0} \mZ$$
    \textbf{Step 3:} Take group cohomology! 
    \begin{align*}
        H^0(C_2;\mZ) & = \frac{\ker(\mZ\xr{0}\mZ)}{\Im(0\to \mZ)} = \mZ\\
        H^1(C_2;\mZ) & = \frac{\ker(\mZ\xr{2}\mZ)}{\Im(\mZ\xr{0} \mZ)} = 0\\
        H^2(C_2;\mZ) & = \frac{\ker(\mZ\xr{0}\mZ)}{\Im(\mZ\xr{2} \mZ)} = \mZ/2\\
        \vdots&
    \end{align*}
    Which ends up resulting in $$H^n(C_2;\mZ) = \begin{cases}
        \mZ & n = 0\\
        \mZ/2 & n>0 \text{ and even}\\
        0 & \text{else}.
    \end{cases}$$
    As a fun little surprise, look up $H^\ast(\mRP^\infty;\mZ)$. Do you notice anything?
\end{example}