\usepackage{multicol,amsmath,amssymb,wasysym,mathtools,graphicx,tikz,geometry,hyperref,amsthm,makeidx,newtx,dsfont,sseq,pagecolor,xcolor,enumerate,pdflscape,spectralsequences,pdfpages}
\usetikzlibrary{cd,automata,matrix,decorations.pathreplacing,decorations.markings}
\usepackage[normalem]{ulem}
\usepackage[breakable]{tcolorbox}


\geometry{margin = 1in}

\definecolor{wongorange}{RGB}{230,159,0}
\definecolor{wonglightblue}{RGB}{86,180,233}
\definecolor{wonggreen}{RGB}{0,158,115}
\definecolor{wongyellow}{RGB}{240,228,66}
\definecolor{wongdarkblue}{RGB}{0,114,178}
\definecolor{wongred}{RGB}{213,94,0}
\definecolor{wongpink}{RGB}{204,121,167}
\definecolor{scottygreenyellow}{RGB}{169,255,23}
\definecolor{vantablack}{RGB}{26,26,26}


% \pagecolor{black}
\pagecolor{vantablack} %Nice gray black
\color{white}


\hypersetup{
    colorlinks=true,
    linkbordercolor={vantablack},
    linkcolor={wonggreen},
    urlcolor = {wonglightblue},
    citecolor = {wongorange}}

\usepackage[textwidth=2cm,textsize=tiny,colorinlistoftodos]{todonotes}
%% S.T. notes
\newcommand{\stnote}[1]{\todo[color=cyan!20!black,linecolor=cyan!20!black,size=\tiny,textcolor = white]{#1}}
\newcommand{\stnoteil}[1]{\ \todo[inline,color=cyan!20!black,linecolor=cyan!20!black,textcolor = white,size=\normalsize]{{#1}}}



\newcommand{\by}{\times}
\newcommand{\oby}{\otimes}
\newcommand{\sets}[1]{\left\{ #1 \right\}}
\newcommand{\abs}[1]{\left| #1 \right|}
\newcommand{\aabs}[1]{\left\| #1 \right\|}
\newcommand{\parenths}[1]{\left( #1 \right)}
\newcommand{\pparenths}[1]{\left(\!\left( #1 \right)\!\right)}
\newcommand{\bracks}[1]{\left\lbrack #1 \right\rbrack}
\newcommand{\bbracks}[1]{\left\lbrack\!\left\lbrack #1 \right\rbrack\!\right\rbrack}
\newcommand{\angles}[1]{\left\langle #1 \right\rangle}
\newcommand{\ol}[1]{\overline{#1}}
\newcommand{\ul}[1]{\underline{#1}}
\renewcommand{\emptyset}{\varnothing} 
\newcommand{\inv}{^{-1}}
\newcommand{\oldphi}{\phi}
\renewcommand{\phi}{\varphi}
\newcommand{\gray}[1]{\textcolor{black!60}{ #1 }}
\newcommand{\red}[1]{\textcolor{red!60!orange}{ #1 }}
\newcommand{\xr}[1]{\xrightarrow{#1}}
\newcommand{\onto}{\twoheadrightarrow}
\newcommand{\into}{\hookrightarrow}
\newcommand{\upto}{\uparrow}
\newcommand{\downto}{\downarrow}
\newcommand{\from}{\leftarrow}
\newcommand{\ds}{\displaystyle}
\newcommand{\eps}{\varepsilon}
\newcommand{\dee}{\partial}
\renewcommand{\tilde}[1]{\widetilde{#1}}
\renewcommand{\hat}[1]{\widehat{#1}}
\newcommand{\bigmid}{~\big|~}
\newcommand{\biggmid}{~\bigg|~}
\newcommand{\Biggmid}{~\Bigg|~}
\newcommand{\acts}{\curvearrowright}
\newcommand{\spa}{\mathsf{span}}
\renewcommand{\implies}{\Rightarrow}
\newcommand{\nond}{\noindent}
\newcommand{\Fr}{\mathsf{Fr}}
\newcommand{\mCP}{\mathbb{CP}}
\newcommand{\mRP}{\mathbb{RP}}
\newcommand{\mHP}{\mathbb{HP}}
\newcommand{\SO}{\mathsf{SO}}
\newcommand{\Spin}{\mathsf{Spin}}
\newcommand{\GL}{\mathsf{GL}}
\newcommand{\lO}{\mathsf{O}}
\newcommand{\Pin}{\mathsf{Pin}}
\newcommand{\mCl}{\mathbb{C}\text{l}}
\newcommand{\Sq}{\text{Sq}}
\newcommand{\cTop}{\mathsf{Top}}
\newcommand{\uno}{\mathds{1}}
\newcommand{\overoverset}[3]{\overset{#1}{\overset{#2}{#3}}}
\newcommand{\underunderset}[3]{\underset{#1}{\underset{#2}{#3}}}
\newcommand{\lspinc}{\text{spin}^c}
\newcommand{\ndee}{{\eth}}
\renewcommand{\Im}{\text{Im }}
\renewcommand{\Re}{\text{Re }}
\newcommand{\Diff}{\text{Diff}}
\newcommand{\K}{\text{K}}
\newcommand{\mtR}{\widetilde{\mR}}
\newcommand{\mtC}{\widetilde{\mathbb{C}}}
\newcommand{\sedots}{\rotatebox{-45}{$\vdots$}}
\newcommand{\stwomatrixsmall}{\begin{psmallmatrix}0&1\\1&0\end{psmallmatrix}}
\newcommand{\stwomatrix}{\begin{pmatrix}0&1\\1&0\end{pmatrix}}

\newcommand{\ortho}{\mathsf{O}}
\newcommand{\unitary}{\mathsf{U}}
\newcommand{\sortho}{\mathsf{SO}}
\newcommand{\sunitary}{\mathsf{SU}}

\DeclareMathOperator{\proj}{\text{proj}}
\DeclareMathOperator{\coker}{\text{coker}}
\DeclareMathOperator{\Ch}{\text{Ch}}
\DeclareMathOperator{\supp}{\text{supp}}
\DeclareMathOperator{\colim}{\text{colim}}
\DeclareMathOperator{\Pic}{\text{Pic}}
\DeclareMathOperator{\Th}{\text{Th}}
\DeclareMathOperator{\ind}{\text{ind}}
\DeclareMathOperator{\Aut}{\text{Aut}}
\DeclareMathOperator{\End}{\text{End}}
\DeclareMathOperator{\Hom}{\text{Hom}}
\DeclareMathOperator{\FBF}{\text{FBF}}
\DeclareMathOperator{\BF}{\text{BF}}
\DeclareMathOperator{\SW}{\text{SW}}
\DeclareMathOperator{\calSW}{\mathcal{SW}}
\DeclareMathOperator{\tSW}{\widetilde{\text{SW}}}
\DeclareMathOperator{\tcalSW}{\widetilde{\mathcal{SW}}}
\DeclareMathOperator{\bSW}{\mathbb{SW}}
\DeclareMathOperator{\tbSW}{\widetilde{\mathbb{SW}}}

\newcommand{\lineyspace}{
  \bigskip
  \hrule
  \bigskip
}
\newcommand{\dirac}{\partial\!\!\!/}
% \setcounter{secnumdepth}{0}




\makeatletter
\newcommand*{\relrelbarsep}{.386ex}
\newcommand*{\relrelbar}{%
  \mathrel{%
    \mathpalette\@relrelbar\relrelbarsep
  }%
}
\newcommand*{\@relrelbar}[2]{%
  \raise#2\hbox to 0pt{$\m@th#1\relbar$\hss}%
  \lower#2\hbox{$\m@th#1\relbar$}%
}
\providecommand*{\rightrightarrowsfill@}{%
  \arrowfill@\relrelbar\relrelbar\rightrightarrows
}
\providecommand*{\leftleftarrowsfill@}{%
  \arrowfill@\leftleftarrows\relrelbar\relrelbar
}
\providecommand*{\xrightrightarrows}[2][]{%
  \ext@arrow 0359\rightrightarrowsfill@{#1}{#2}%
}
\providecommand*{\xleftleftarrows}[2][]{%
  \ext@arrow 3095\leftleftarrowsfill@{#1}{#2}%
}
\makeatother

\makeatletter
\@tfor\next:=ABCDEFGHIJKLMNOPQRSTUVWXYZ\do{%
  \def\command@factory#1{%
    \expandafter\def\csname cal#1\endcsname{\mathcal{#1}}
  }
 \expandafter\command@factory\next
}
\makeatletter
\@tfor\next:=ABCDEFGHIJKLMNOPQRSTUVWXYZ\do{%
  \def\command@factory#1{%
    \expandafter\def\csname m#1\endcsname{\mathbb{#1}}
  }
 \expandafter\command@factory\next
}
\makeatletter
\@tfor\next:=ABCDEFGHIJKLMNOPQRSTUVWXYZabcdefghijklmnopqrstuvwxyz\do{%
  \def\command@factory#1{%
    \expandafter\def\csname fr#1\endcsname{\mathfrak{#1}}
  }
 \expandafter\command@factory\next
}




\tcbuselibrary{theorems}
\newtcbtheorem{theorem}{Theorem}{breakable,colback = black!80!green, colframe=green!35!black,fonttitle=\bfseries,coltext=white}{thm}
\newtcbtheorem{corollary}{Corollary}{breakable,colback = black!82!green,colframe=green!30!black,fonttitle=\bfseries,coltext = white}{cor}
\newtcbtheorem{lemma}{Lemma}{breakable,colback = black!85!green,colframe=green!25!black,fonttitle=\bfseries,coltext = white}{lem}
\newtcbtheorem{proposition}{Proposition}{breakable,colback = green!30!black,colframe=green!20!black,fonttitle=\bfseries,coltext = white}{lem}
\newtcbtheorem{definition}{Definition}{breakable,colback = black!80!blue,colframe=blue!35!black,fonttitle=\bfseries,coltext = white}{def}
\newtcbtheorem{exercise}{Exercise}{breakable,colback = black!80!yellow,colframe=yellow!35!black,fonttitle=\bfseries,coltext=white}{exercise} 
\newtcbtheorem{remark}{Remark}{breakable,colback = black!80!wongpink,colframe=wongpink!35!black,fonttitle=\bfseries,coltext=white}{exercise} 
\newtcbtheorem{problem}{Problem}{breakable,colback = black!80!cyan,colframe=cyan!35!black,fonttitle=\bfseries,coltext=white}{problem} 
\newtcbtheorem{example}{Example}{breakable,colback = black!80!wongpink,colframe=wongpink!35!black,fonttitle=\bfseries,coltext=white}{exmple}
\newtcbtheorem{question}{Question}{breakable,colback = black!80!wongpink,colframe=wongpink!35!black,fonttitle=\bfseries,coltext=white}{qn}
\newtcbtheorem{conjecture}{Conjecture}{breakable,colback = black!80!wongpink,colframe=wongpink!35!black,fonttitle=\bfseries,coltext=white}{conj}
\newtcbtheorem{claim}{Claim}{breakable,colback = black!80!wongpink,colframe=wongpink!35!black,fonttitle=\bfseries,coltext=white}{claim}
\newtcbtheorem{fact}{FACT}{breakable,colback = black!80!wongorange,colframe=wongorange!35!black,fonttitle=\bfseries,coltext=white}{fact}

\newtcbox{\tcmathtex}{on line,colback=vantablack!40!wonglightblue,colframe=vantablack,size=fbox,arc=3pt,boxrule=0.8pt,coltext=  white}
\newcommand{\tcmtblue}[1]{\tcmathtex{$#1$}}

\newcommand{\alerttext}[1]{\textcolor{wongorange}{\textbf{#1}}}




% \pagestyle{empty}
\setcounter{secnumdepth}{0}
\makeindex