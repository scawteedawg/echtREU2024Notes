\appendix
\newcommand{\defindex}[1]{#1}
\newcommand{\scottycomment}[1]{\footnote{#1}}
\begin{definition}{pullbacks}{pullback}
    Let $\calC$ be a category which contains the diagram \begin{center}
      \begin{tikzcd}[ampersand replacement = \&]
        \&Y\ar[d,"g"]\\
        X\ar[r,"f"]\& Z
      \end{tikzcd}.
    \end{center}
    A \defindex{pullback} of this diagram $W$ is 3 pieces of information
    \begin{itemize}
        \item An object $\textcolor{wonglightblue}{W}\in \text{ob}(\calC)$
        \item A map $\textcolor{wonglightblue}{W\xr{p_1}}X$
        \item A map $\textcolor{wonglightblue}{W\xr{p_2}}Y$
    \end{itemize} such that 
    \begin{itemize}
        \item[-] The diagram   \begin{tikzcd}[ampersand replacement = \&]
            \textcolor{wonglightblue}{W}\ar[r,wonglightblue,"p_2"]\ar[d,wonglightblue,"p_1"']\ar[dr,phantom,"\circlearrowright"]\&Y\ar[d,"g"]\\
            X\ar[r,"f"]\& Z 
          \end{tikzcd} commutes \scottycomment{``$\circlearrowright$'' is a `long'hand for commutes, and people usually suppress it from notation.}
          \item[-] If someone hands you a commutative diagram \begin{tikzcd}[ampersand replacement = \&]
            \textcolor{wongpink}{A}\ar[r,wongpink,"h_2"]\ar[d,wongpink,"h_1"']\ar[dr,phantom,"\circlearrowright"]\&Y\ar[d,"g"]\\
            X\ar[r,"f"]\& Z 
          \end{tikzcd}, then there is a \emph{UNIQUE} map $\tilde{h}$ such that \begin{tikzcd}[ampersand replacement = \&,background color = wonglightblue!5]
            \textcolor{wongpink}{A}\ar[drr,bend left = 20,"h_2",wongpink]\ar[ddr,phantom,"\circlearrowright"]\ar[drr,phantom,"\circlearrowright"]\ar[ddr,wongpink,bend right = 20,"h_1"']\ar[dr,thick,wonggreen,"{\tilde{h}}"]\&\&\\
            \&\textcolor{wonglightblue}{W}\ar[r,wonglightblue,"p_2"]\ar[d,wonglightblue,"p_1"']\ar[dr,phantom,"\circlearrowright"]\&Y\ar[d,"g"]\\
            \&X\ar[r,"f"]\& Z 
          \end{tikzcd}.
    \end{itemize}
    \scottycomment{As a shorthand, people usually write the pullback like this: 
    \begin{center}
      \begin{tikzcd}[ampersand replacement = \&]
        W\ar[r]\ar[d]\ar[dr,phantom,"\lrcorner",very near start]\& Y\ar[d]\\
        X\ar[r]\&Z\\
      \end{tikzcd}
    \end{center}}

    \medskip
    Let \textcolor{wongorange}{$\calC$} be a category with a subcategory \textcolor{wonglightblue}{$\calD$}. We say \textcolor{wonglightblue}{$\calD$} is \defindex{closed under pullbacks} by morphisms in \textcolor{wongorange}{$\calC$} if for all arrows \textcolor{wonglightblue}{$X\xr{f}Z$} in $\textcolor{wonglightblue}{\calD}$ and for all $\textcolor{wongorange}{Y\xr{g}Z}$ in \textcolor{wongorange}{$\calC$} such that we can form the pullback   \begin{tikzcd}[ampersand replacement = \&]
        W\ar[r,"p_1",wonglightblue]\ar[d,"p_2"]\ar[dr,phantom,"\lrcorner",very near start]\& Y\ar[d,"g",]\\
        X\ar[r,"f"]\& Z
      \end{tikzcd}, then the arrow \textcolor{wonglightblue}{$W\xr{p_1}Z$} is in \textcolor{wonglightblue}{$\calD$}.
\end{definition}

\begin{definition}{pushouts}{pushout}
    Let $\calC$ be a category which contains the diagram \begin{center}
      \begin{tikzcd}[ampersand replacement = \&]
        Z\&Y\ar[from =l,"g"]\\
        X\ar[from = u,"f"]\& 
      \end{tikzcd}.
    \end{center}
    A \defindex{pushout} of this diagram $W$ is 3 pieces of information
    \begin{itemize}
        \item An object $\textcolor{wonglightblue}{W}\in \text{ob}(\calC)$
        \item A map $X\textcolor{wonglightblue}{\xr{i_1}W}$
        \item A map $Y\textcolor{wonglightblue}{\xr{i_2}W}$
    \end{itemize} such that 
    \begin{itemize}
        \item[-] The diagram   \begin{tikzcd}[ampersand replacement = \&]
            Z\&Y\ar[from =l,"g"]\ar[d,wonglightblue,"i_2"]\\
            X\ar[from = u,"f"]\ar[r,wonglightblue,"i_1"]\& \textcolor{wonglightblue}{W}
          \end{tikzcd} commutes 
          \item[-] If someone hands you a commutative diagram \begin{tikzcd}[ampersand replacement = \&]
            Z\&Y\ar[from =l,"g"]\ar[d,wongpink,"\ell_2"]\\
            X\ar[from = u,"f"]\ar[r,wongpink,"\ell_1"]\& \textcolor{wongpink}{A}
          \end{tikzcd}, then there is a \emph{UNIQUE} map $\tilde{\ell}$ such that \begin{tikzcd}[ampersand replacement = \&,background color = wonglightblue!5]
            Z\&Y\ar[from =l,"g"]\ar[d,wonglightblue,"i_2"]\ar[ddr,bend left = 20,"\ell_2",wongpink]\&\\
            X\ar[from = u,"f"]\ar[r,wonglightblue,"i_1"]\ar[drr,bend right = 20,"\ell_1"',wongpink]\& \textcolor{wonglightblue}{W}\ar[dr,wonggreen,"\tilde{\ell}",thick]\&\\
            \&\& \textcolor{wongpink}{A}
          \end{tikzcd}.
    \end{itemize}
    \scottycomment{As a shorthand, people usually write the pushout like this: 
    \begin{center}
        \begin{tikzcd}[ampersand replacement = \&]
            Z\ar[dr,"\ulcorner",very near end,phantom]\&Y\ar[from =l,"g"]\ar[d,"i_2"]\\
            X\ar[from = u,"f"]\ar[r,"i_1"]\& {W}
          \end{tikzcd}
    \end{center}}
\end{definition}