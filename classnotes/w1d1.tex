\section{Day 1}

\subsection{Lecture 1 (Irina Bobkova): Lot's of definitions!}
\begin{tcolorbox}[colback = black,coltext = white]{Plan}{}

    We are going to try and compute $$\pi_\ast(E^{hC_6}\wedge V(0)).$$
\end{tcolorbox}
Let's define a few things. 
\begin{itemize}
    \item $C_6$ is a cyclic group of order 6.
    \item $E^{hC_6}$ is a Morava $E$-theory and this is a spectrum (think a space).
    \item $E(n,p)$ has $n$ the {chromatic height} and $p$ a {prime}.
    \item $G\acts$ on sets, spaces, or spectra. 
    \item Let $S$ be a space with a $G$-action. \index{fixed points}\index{fixed points!homotopy}
    \begin{align*}
        S^G&=\sets{s\in S\mid g\cdot s = s~\forall g\in G}& E^{hC_6}&:=\sets{\text{homotopy $G$ fixed points}}\\
        & = \sets{G\text{-fixed points of }S}
    \end{align*}
    \item $X\wedge Y$ is the {smash product}\index{smash product} of $X,Y$ and is defined to be $$X\wedge Y:= \frac{X\by Y}{X\vee Y}.$$
    \item $V(0):=\mS/2$ the {Moore space}. Take a sphere $S^n$, and consider the degree map $S^n\xr{m}S^n$. Here is an instance of this map. $$S^n\xr{2}S^n\vee S^n\to S^n.$$ 
    \begin{center}
        \begin{tikzpicture}
            \draw[vantablack,shading = ball] (0,0) circle (1);
            \draw[->] (1.1,0) -- (1.9,0);
            \draw[vantablack,shading = ball] (2.5,.5) circle (.5);
            \draw[vantablack,shading = ball] (2.5,-.5) circle (.5);
            \draw[blue!80!black,fill = blue!80!black] (2.5,0) circle (.05);
            \draw [->] (3.1,0)--(3.9,0);
            \draw[vantablack,shading = ball] (5,0) circle (1);
            \draw (1.5,.2) node {\tiny pinch};
            \draw (3.5,.2) node {\tiny fold};
        \end{tikzpicture}
    \end{center}
    The thing to take away is that for a degree $m$-map between $n$-spheres, you can create this map as a composition $$S^n\xr{\text{pinch}}\bigvee_1^m S^n\xr{\text{fold}}S^n$$ to get a degree $m$ map. More details about this can be found in \cite[\S 2.2]{Hatcher}
    \item The {sphere spectrum}\index{spectrum!sphere spectrum} is a topological object which can be written as $$\mS:=\sets{S^0,S^1,S^2,\dots}.$$ FACT: We can define a degree $m$ map on the sphere spectrum. 
    \item Fiber/cofiber sequences:\index{fiber sequence}\index{cofiber sequence}. In spectra, fiber and cofiber sequences are the same! This is an anolog of a short exact sequence for groups. Here's an example. Consider the map 
    \begin{center}
      \begin{tikzcd}[ampersand replacement = \&]
        \mZ\ar[r,"\by 2"]\& \mZ \\ [-20pt]
        0\ar[r,mapsto] \& 0\\[-20pt]
        1\ar[r,mapsto]\& 2.
      \end{tikzcd}
    \end{center}
    The kernel of this map is 0! The cokernel of this map is $\mZ/2$. This gives a short exact sequence of groups $$0\to \mZ\xhookrightarrow{\by 2}\mZ\onto \mZ/2\to 0.$$ We can do an analog with spectra to get $$\mS\xr{2}\mS\to \underbrace{V(0)}_{\text{cofiber}(2)}\to \Sigma \mS\xr{\Sigma 2}\Sigma \mS\to \Sigma V(0)\to \cdots.$$

    Note: there is a way to understand fibers and cofibers as pushout and pullback diagrams. 
    \item For spaces $\Sigma$, aka reduced suspension\index{suspension!reduced}, exists for all $n\in \mN$; you can suspend a space however many times you want, $\Sigma^n$. In spectra-land, you can \emph{negatively}-suspend a space, aka desuspend the space, i.e. you can do $\Sigma^n$ for all $n\in \mZ$. 
    \item $\ds\pi_\ast = \bigoplus_{i\in \mZ} \pi_i$. Here \begin{align*}
        \pi_n(X)&:=\text{Maps}(S^n,X)_{/\text{homotopy}}.
    \end{align*} Sometimes we write this as $[S^n,X]$ so we have to type less!
    \item Let $X$ be a space, and let $f\in \pi_n(X), g\in \pi_m(X)$, meaning that we have $$f:S^n\to X, \quad g:S^m\to X.$$ What is $f\cdot g$ if we're talking about $\pi_\ast$ having a ``ring structure.'' Then we have 
    \begin{center}
      \begin{tikzcd}[ampersand replacement = \&]
        \& X\wedge S^m\ar[dr,"\uno \by g"]\&\\
        S^{n+m}= S^n\wedge S^m\ar[ur,"f\wedge \uno"]\ar[dr,"\uno \wedge g"']\ar[rr,"f\wedge g"]\& \&X\wedge X\ar[r,"\mu"]\& X\\
        \& S^n\wedge X\ar[ur,"f\wedge \uno"']
      \end{tikzcd}
    \end{center}
    which gives us a map $\pi_{n+m}(X\wedge X)$. If we have a map $X\wedge X \xr{\mu} X$, then we're good; this is an honest to goodness ring! An instance of this is $S^0$. Try it out! For us $V(0) = \text{Cofiber}(2)$ is not a ring. 
\end{itemize}


\newpage

\subsection{Spectra}

\begin{definition}{Spectrum}{}
    A \defindex{spectrum}\footnote{What we describe here is sometimes referred to as a prespectrum. Some people require a spectrum to have the structure maps as $X_n\to \Omega X_{n+1}$ and homeomorphisms.} $X$ is a collection of pointed spaces $$\sets{X_0,X_1,X_2,\dots} = \sets{X_n}_{n\in \mN}$$ together with structure maps $$\Sigma X_n\to X_{n+1}.$$
\end{definition}
\begin{example}{}{}
    \begin{enumerate}
        \item The sphere spectrum $\mS = \sets{S^0,S^1,\dots}$ and homeomorphisms $\Sigma S^n\xr{\cong}S^{n+1}$. 
        \item Suspension spectrum $\Sigma^\infty X = \sets{X,\Sigma X, \Sigma^2 X,\dots}$ \index{spectrum!suspension spectrum} with structure maps $$\Sigma (\Sigma^\infty X)_n = \Sigma \Sigma^n X\xr{\cong }\Sigma^{n+1}X = (\Sigma^\infty X)_{n+1}.$$
        \item For some (non-suspension) spectra, we can describe the spaces, but for the majority of spectra, we cannot. 
    \end{enumerate}
\end{example}
\begin{center}
    \begin{tikzcd}[ampersand replacement = \&,column sep =2cm]
      \mathsf{Top}_\ast\ar[r,shift left = 5pt,"\Sigma^\infty"]\ar[rr,bend left = 50,shift left = 5pt,"\Sigma^\infty"]\& \mathsf{PreSp}\ar[l,shift left = 5pt,"\Omega^\infty"] \ar[r,shift left = 5pt,"L"]\& \mathsf{Sp}\ar[l,shift left = 5pt, "U"]\ar[ll, bend left = 50,shift left = 5pt,"\Omega^\infty"]
    \end{tikzcd}

    \begin{center}
      \begin{tikzcd}[ampersand replacement = \&]
        \mathsf{PreSp}\ar[r,"L"]\& \mathsf{Sp} \\ [-20pt]
        (X_n)_{n\in \mN}\ar[r,mapsto] \& (LX_n)_{n\in \mN}, \& LX_n:= \colim\parenths{X_n\into \Omega X_{n+1}\into \Omega^2 X_{n+t}\into \cdots}
      \end{tikzcd}

      \begin{tikzcd}[ampersand replacement = \&]
        \mathsf{Sp}\ar[r,"\Omega^\infty"]\& \mathsf{Top}_\ast \\ [-20pt]
        (X_n)_{n\in \mN}\ar[r,mapsto] \& X_0. \& \hspace{2.5in}~
      \end{tikzcd}
    \end{center}
  \end{center}

``Why were spectra invented?'' you may ask. One answer comes in the form of Brown's representability theorem. To understand this, we need a few definitions. 
\begin{definition}{}{}
    A \defindex{generalized homology} theory $E$ is a functor $$E:\mathsf{Spaces}\to \mathsf{GradedAbGrps}$$ with the properties 
    \begin{itemize}
        \item Homotopy: Homotopic spaces have the same homology.
        \item Exactness: Exact sequence in homology from a cofiber sequence. 
        \item Excision: If $X = A\cup B$, then $E_\ast(A,A\cap B)\to E_\ast(X,B)$ is an isomorphism. 
        \item Additivity: Coproducts in Spaces induce coproducts in homology.
    \end{itemize}
    For more details, see \href{https://en.wikipedia.org/wiki/Cohomology#Axioms_and_generalized_cohomology_theories}{Wikipedia on generalized cohomology}.
\end{definition}

\begin{theorem}{Brown's representability Theorem}{}
    There is an isomorphism between generalized (co)homology theories and spectra.\index{Brown's representability Theorem} Given a spectrum $\calE$, the homology is given by $$\calE_\ast(X) = \pi_\ast(\calE\wedge X).$$ The cohomology associated to the spectrum $\calE$ is given by $$\calE^\ast(X) = [X,\calE].$$
\end{theorem}

\begin{definition}{Fiber Sequences}{}
    We'll come back to this! The key is that in spectra land, it goes back and forth in both directions. 
\end{definition}

\begin{fact}{}{}
    Any fiber sequence $X\to Y\onto Z$ gives rise to a long exact sequence in $\pi_\ast$, %$$\cdots \to \pi_{k+1}Z\to \pi_k X\to \pi_k Y\to \pi_k Z\to \pi_{k-1}X\to \cdots$$
    \begin{center}
      \begin{tikzcd}[ampersand replacement = \&]
        \&\cdots\ar[r]\& \pi_{k+1}Z\ar[dll,out = 225,in = 45]\\ 
        \pi_k X\ar[r]\&\pi_k Y\ar[r]\&\pi_k Z\ar[dll,out = 225,in = 45]\\ 
        \pi_{k-1}X\ar[r]\& \cdots
      \end{tikzcd}
    \end{center}
\end{fact}

\newpage

\subsection{Lecture 2 (Jack Carlisle): Algebra}

\subsubsection{Rings}
First, let's talk about commutative \emph{(order of multiplication doesn't matter)}, unital \emph{(the ring has the element 1)} rings.\index{ring} Every time I write $R$ as a ring, I mean this version of a ring. 
\begin{example}{}{}
    \begin{itemize}
        \item $\mZ$
        \item $\mZ/n$ for $n\geq 2$
        \item $\mF_p:=\mZ/p$ with $p$ a prime. A special case of this is $\mF_2 = (\sets{0,1},+,\times)$. 
        \item $\mZ[x]$, $\mF_2[x]$, $R[x]$, aka polynomial rings in one variable. 
        \item $\frac{\mF_2[x]}{(x^3+1)}$, a ring mod out by an ideal. 
        \item $\mZ[G]$ for $G$ an abelian group, the group ring. 
        \item $\ds\mF_4:=\frac{\mF_2[x]}{(x^2+x+1)}$, \ul{the} field with 4($=2^2$) elements. 
        \item $\mZ[[x]] = \sets{\sum_0^\infty a_k x^k\mid \forall k, a_k\in \mZ}$, the power series ring
        \item $\mZ((x))$, the Laurent series ring. 
    \end{itemize}
\end{example}
\subsubsection{Modules}
\begin{definition}{Module}
    A \defindex{module} $M$ over a commutative ring $R$ is an abelian group $M$ together with a scaling map $$R\oby M\to M$$ $$r\oby m\mapsto r\cdot m.$$
\end{definition}
\begin{example}{}{}
    A vector space $V$ over the ring $\mR$ (or any field $\mF$) is the same thing as an $\mR$-module. 
\end{example}
\begin{example}{}{}
    If $R$ is a ring, then an ideal $I\subseteq R$ is the same thing as a submodule of $R$.
\end{example}
\subsubsection{Exact Sequences}
\begin{definition}{Short Exact Sequence}{}
    A \defindex{short exact sequence} is $$0\to M'\xr{f}M\xr{g}M''\to 0,$$ such that $$\ker(\text{each map}) = \Im(\text{previous map}).$$ For specificity, we need 
    \begin{enumerate}[i.]
        \item $f$ is injective
        \item $g$ is surjectve
        \item $\ker g = \Im f$. 
    \end{enumerate}
\end{definition}
\begin{example}{}{}
    Let $R$ be a field, say $\mF_2$, let $V$ be an $R$-vector space and let $W\leq V$ be a subspace. Then $$0\to W\to V\to V/W\to 0$$ is a short exact sequence. 
\end{example}
\begin{example}{}{}
    Let $R =\mZ$. Consider the map $$0\to \mZ/2\xr{[1~0]}\mZ/2\oplus \mZ/2\xr{\begin{bsmallmatrix}
        0 & 0\\ 0 & 1 
    \end{bsmallmatrix}}\mZ/2\to 0.$$ What is the composition of these maps? Is this sequence exact?
\end{example}
\begin{example}{}{}
    Let $R = \mZ$. Then consider the sequence $$0\to \mZ/2\xr{(-)\cdot 2}\mZ/4\xr{()\mod 2}\mZ/2\to 0.$$ Is this a short exact sequence? If so, how does it compare to the prior example?
\end{example}
\begin{remark}{}{}
    $\mZ/2\oplus \mZ/2\neq \mZ/4$ as groups. Prove it!
\end{remark}
Oftentimes, we are interested in some module $M$, and we know that it fits into a short exact sequence $$0\to M'\to M\to M''\to 0$$ where $M',M''$ are known. Determining $M$ given $M'$ and $M''$ is called an extension problem.

\begin{definition}{$p$-adics}{}
    Another ring of interest is the \defindex{$p$-adic integers $\mZ_p$} also denoted $\mZ_p^\wedge$ where $\wedge$ means completed. Another way to write this is \begin{align*}\mZ_p = \mZ_p^\wedge &= \varprojlim \mZ/p^k\\ &= \varprojlim(\cdots \to \mZ/p^3\to \mZ/p^2\to\mZ/p\to 0)\\ & = \sets{(a_1,a_2,\dots)\mid a_i \in \mZ/p^i, \quad a_{i+1}\equiv a_i \mod p^i}.\end{align*}
\end{definition}

