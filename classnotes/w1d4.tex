\section{Day 4}
\subsection{Lecture 1}

\noindent \textit{Disclaimer: today had a lot of spectral sequences, and I'm not sure how coherent the writing is below here. I tried to get the spectral sequences correct, but I may have missed some important language helping us make these.}

Yesterday we computed a homotopy fixed point spectral sequence. Suppose that $C_2\acts\calE$ where $\calE$ is a spectrum. 

$$E_2^{s,t} = H^s(C_2;\pi_t \calE)\Rightarrow \pi_{t-s}\calE^{hC_2}.$$

\DeclareSseqGroup\etatowertwo{}{
    \class[rectangle](0,0)
    \foreach\i in {1,...,11}{
        \class[fill = white](\i,\i)
        \structline(\i-1,\i-1,-1)(\i,\i,-1)
    }
}
\begin{sseqdata}[name = HFPSSd2,x range ={0}{16},y range = {0}{4},xscale = .5,Adams grading]
    \foreach \k in {-2,...,4}{
        \etatowertwo({4*\k},0)
    }
    \d[wongorange]3 (-4,0)
    \replacesource[rectangle]
    \d[wongorange]3(4,0)
    \replacesource[rectangle]
    \d[wongorange]3(12,0)
    \replacesource[rectangle]
    \foreach \n in {1,...,6}{
        \d[wongorange]3 ({-4+\n},\n)
        \d[wongorange]3({4+\n},\n)
        \d[wongorange]3({12+\n},\n)
    }
\end{sseqdata}
\begin{center}
    \printpage[name = HFPSSd2]

    \printpage[name = HFPSSd2,page = 4]
\end{center}

The final $E_\infty$ page is $\pi_\ast E^{hC_2}$. This gives $\pi_\ast E = \mZ[u^{\pm 1}]$ where $\abs{u} =2$. (If this were a polynomial ring where $x$ is a variable, then $u$ would have the same weight the $x^2$ ``part.'')

\begin{question}{}{}
    How do we compute $\pi_\ast(E^{hC_2}\wedge V(0))$. 
\end{question}
There is a spectral sequence $$\underbrace{E_2^{s,t} = H^s(C_2,\pi_t(E\wedge V(0)))}_{\text{Question 1}}\Rightarrow \underbrace{\pi_{t-s}E^{hC_2}\wedge V(0)}_{\text{Question 2}}$$

For Question 1, we have 
\begin{align*}
    \mS&\xr{2}\mS\to V(0)\tag{fiber sequence}\\
    E\wedge \mS&\xr{2}E\wedge \mS \to E\wedge V(0)\tag{fiber sequence}
\end{align*}
This induces a long exact sequence in homotopy groups. 
\begin{center}
  \begin{tikzcd}[ampersand replacement = \&]
    \&\cdots\ar[r]\&\pi_{k+1}(E\wedge V(0))\ar[dll,out = 225,in = 45]\\ 
    \pi_k(E\wedge \mS)\ar[r,"2"]\& \pi_k(E\wedge \mS)\ar[r]\& \pi_k(E\wedge \mS)\ar[dll,out = 225,in = 45]\\ 
    \pi_{k-1}(E\wedge \mS)\ar[r,"2"]\& \cdots.
  \end{tikzcd}
\end{center}
After following the LES (using our spectral sequence $E_\infty$ page on the the previous page), we get $$\pi_t(E\wedge V(0)) = \begin{cases}
    \mZ/2 & t\text{ even}\\
    0 & t\text{ odd}
\end{cases}.$$
Therefore, our $E_2^{s,t}$ group we were looking for is coming from $$H^s(C_2,\pi_t(E\wedge V(0))).$$

\begin{center}
    \begin{tikzcd}[ampersand replacement = \&]
      \&\cdots\ar[r]\&\pi_{3}(E\wedge V(0))\ar[dll,out = 225,in = 45]\\ 
      \pi_2(E\wedge \mS)\ar[r,"2","u\mapsto 2u"']\& \pi_2(E\wedge \mS)\ar[r]\& \pi_2(E\wedge \mS)\ar[dll,out = 225,in = 45]\\ 
      \pi_{1}(E\wedge \mS)\ar[r,"2"]\& \pi_1(E\wedge \mS)\ar[r]\& \pi_1(E\wedge \mS)\ar[dll,out = 225,in = 45]\\ 
      \pi_{0}(E\wedge \mS)\ar[r,"2"]\& \pi_0(E\wedge \mS)
    \end{tikzcd}
\end{center}

Here, we need to find $H^s(C_2;\pi_2(E\wedge V(0)))$ and put this into an Adams grading. This means the $x$-axis doesn't represent $t$ but rather represents $t-s$.  When you do this computation, we get the following picture for the $E_2$ page. 
\DeclareSseqGroup\tenamwednesday{}{
    \foreach\k in {0,4}{
        \foreach \j in {0,...,6}{
            \class[fill = white]({\k+\j},\j)
        }
    }
    \foreach\k in {2,6}{
        \foreach \j in {0,...,6}{
            \class[circlen=2,wongpink,fill = wongpink]({\k+\j},\j)
        }
    }
    }

\begin{sseqdata}[name = EwedgeV0,Adams grading]
        \tenamwednesday(0,0)
        \tenamwednesday(-8,0)
        \tenamwednesday(8,0)
\end{sseqdata}
\begin{center}
    \printpage[name=EwedgeV0,x range = {0}{16}, y range = {0}{6},xscale =.75,grid = chess]
\end{center}
The pink dot represents the module $\mZ/2\sets{u}$ and the white dot represents the module $\mZ/2\sets{1}$. The map $E\xr{\by 2}E\to E\wedge V(0)$ as maps of spectra induces a long exact sequence in $\pi_\ast$ which in this case is a short exact sequence! $$0\to \pi_{2t}E\to \pi_{2t}E\to \pi_{2t}E\wedge V(0)\to 0$$ $$0\to \pi_{\ast}E\to \pi_{\ast}E\to \pi_{\ast}E\wedge V(0)\to 0.$$
\begin{fact}{}{}
    A short exact sequence of modules $$0\to M'\to M\to M''\to 0$$ induces a long exact sequence in group cohomology, $$\cdots \to H^{s-1}(G;M'')\to H^s(G;M')\to H^s(G;M)\to H^s(G;M'')\to H^{s+1}(G;M')\to \cdots.$$
\end{fact}
Now, since we have the ses in modules $$0\to \pi_\ast(E)\to\pi_\ast(E)\to\pi_\ast(E\wedge V(0))\to 0,$$ we get the LES in group cohomology $$\cdots \to H^s(G;\pi_\ast E)\to \underbrace{H^s(G;\pi_\ast E)}_{E_2^{s,\ast}(E)}\to \underbrace{H^s(G,\pi_\ast E\wedge V(0))}_{E_2^{s,\ast}(E\wedge V(0))}\to \textcolor{wongpink}{\underbrace{H^{s+1}(G;\pi_{\ast}E)}_{E_2^{s+1,\ast}(E)}}\to \cdots.$$ Since the elements are inside of our spectral sequences, we actually get maps between spectral sequences! This now tells us information about our differentials between the different spectral sequences!