\section*{Solutions}

\noindent Exercise 1: 

Consider $\mZ/m\xr{\cdot n}\mZ/mn$ and then $\mZ/mn\xr{(-)n}\mZ/n$. This is exact!

\lineyspace

\noindent Exercise 2:

From exercise 3, we have that $V\cong V'\oplus V''$. This means that $\dim V = \dim V' + \dim V''$. 

\lineyspace

\noindent Exercise 3: 

Let $(e_n')_n$ be a basis for $V'$. Let $(f_i'')_i$ be a basis for $V''$. Since $V'\to V$ is injective, let $(e_n)_n$ be a basis for the subspace $V'\subset V$. Note that $V\to V''$ is surjective, so make a choice of $(f_i)_i$ in $V$ which are mapped to $f_i''$ for each $i$. We made a choice here, but this gives us a basis for $V = V'\oplus V''$. 

\lineyspace

\noindent Exercise 4: 

Exercise 1! These are all $\mZ$-modules. 

\lineyspace

\noindent Exercise 5:

$1\implies 2$. $M \cong M'\oplus M''\xr{\proj_2}M''\xr{0\oplus \uno} M'\oplus M'' \cong M$. The composition $M''\xr{0\oplus \uno} M'\oplus M'' \cong M$ is the right map here. 

$2\implies 3$. 

$3\implies 1$.

\lineyspace

\noindent Exercise 6:

\lineyspace

\noindent Exercise 7:

These are all abelian groups and hence $\mZ$-modules, so we need to see where generators are sent since $0$ is sent to $0$ in all of these groups.

\begin{enumerate}
    \item $\mZ\to \mF_2$. $\mZ$ is generated by 1, so there are only two possible maps $1\mapsto 0$ or $1\mapsto 1$. These are the only possibilities.
    \item Same argument as above. 2 maps. 
    \item Here there are 4 maps. 
    \item Here there are only 2. 
    \item Here there are 2 maps since $\mF_4$ only has one element of order 2. 
    \item There are 2 since every map $\mZ_2\to \mF_2$ must factor through $\mZ/2\to \mF_2$ and there are only two maps like this. 
\end{enumerate}

\lineyspace
\noindent Exercise 8: 

\begin{align*}
    H_0(C)& = \frac{\ker(\mZ\to 0)}{\Im(\mZ\xr{2}\mZ)} = \mZ/2&
H_1(C)& = \frac{\ker(\mZ\xr{2}\mZ)}{\Im(\mZ\xr{0}\mZ)} = 0\\
H_2(C)& = \frac{\ker(\mZ\xr{0}\mZ)}{\Im(\mZ\xr{2}\mZ)} = \mZ/2&
H_3(C)& = \frac{\ker(\mZ\xr{2}\mZ)}{\Im(\mZ/4\xr{0}\mZ)} = 0\\
    H_4(C)& = \frac{\ker(\mZ/4\xr{0}\mZ)}{\Im(\mZ/2\xr{2}\mZ/4)} = \mZ/2&
    H_5(C)&= \frac{\ker (\mZ/2\xr{2}\mZ/4)}{\Im(0\to \mZ/2)}= 0
\end{align*}

\lineyspace 

\noindent Exercise 9: 

\begin{align*}
    H_0(C)& = \frac{\ker(\mZ\to 0)}{\Im(\mZ\xr{2}\mZ)} = \mZ/2 &
H_1(C)& = \frac{\ker(\mZ\xr{2}\mZ)}{\Im(\mZ\xr{0}\mZ)} = 0\\
H_2(C)& = \frac{\ker(\mZ\xr{0}\mZ)}{\Im(\mZ\xr{2}\mZ)} = \mZ/2 & 
H_3(C)& = \frac{\ker(\mZ\xr{2}\mZ)}{\Im(\mZ\xr{0}\mZ)} = 0\\
    H_4(C)& = \frac{\ker(\mZ\xr{2}\mZ)}{\Im(\mZ\xr{2}\mZ)} = \mZ/2.
\end{align*}

\lineyspace

\noindent Exercise 10: 

Suppose $f\simeq g$, i.e that there is a map $h:C_i\to D_{i+1}$ such that $f-g = dh+hd$. In a diagram, these are all of the relevant maps. This picture is not commutative. 
\begin{center}
  \begin{tikzcd}[ampersand replacement = \&]
    C_{i+1}\ar[d,"{f,g}"']\ar[r,"d"]\& C_i\ar[r,"d"]\ar[d,"{f,g}"']\ar[dl,"h"']\& C_{i-1}\ar[d,"{f,g}"]\ar[dl,"h"']\\
    D_{i+1}\ar[r,"d"]\& D_i\ar[r,"d"]\& D_{i-1}
  \end{tikzcd}
\end{center}

Let $c\in H_i(C)\setminus\sets{0}$, meaning that $d(c) = 0$ and $c\neq d(c')$ for some other $c'\in C_{i+1}$. We know that $f(c)-g(c) = dh(c)-hd(c)$, and we know that $d(c) = 0$, so $f(c)-g(c) = dh(c)$, or rather $f(c) = g(c)+ dh(c)$. Note that $dh(c)\in |Im(d)$, and so this element is sent to 0 once we go to homology. Therefore, going to homology, we get $H_\ast(f)(c) = H_\ast(g)(c)$. 

\lineyspace

\noindent Exercise 11: 

\lineyspace

\noindent Exercise 12:

