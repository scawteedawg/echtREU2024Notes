\chapter{Appendix}
\newpage
\section{Appendix A}
\begin{definition}{(Homotopy) Fixed points}{}
    Let $G\acts X$ where $X$ is a topological space. The \defindex{fixed points} of $X$ are, all equivalently, \begin{align*}
      X^G&:=\sets{x\in X\mid gx = x~\forall g\in G}\\
      &:=G\mathsf{Maps}(\ast,X)\\
      &:=\mathsf{Maps}(\ast,X)^G = \sets{f:\ast\to X\mid gf(g\inv \ast) = f(\ast)~\forall g\in G}.
    \end{align*}
    The \textbf{homotopy fixed points}\index{fixed points!homotopy} of the topological space $X$ are, all equivalently, 
    \begin{align*}
      X^{hG}&:=G\mathsf{Maps}(EG,X)\\
      &:=\mathsf{Maps}(EG,X)^G = \sets{f:EG\to X\mid ~gf(g\inv e) = f(e)\forall g\in G}\\
      & = \mathsf{Maps}(BG,X).
    \end{align*}
    Here $BG$ is the \defindex{classifying space} for $G$ and $EG$ is the total space of the classifying space. Key properties of $EG$ are that $EG\simeq \ast$, i.e. $EG$ is contractible, and $EG$ has a $G$ action which is free on $EG$. $BG$ is defined to be the quotient space $$BG:=EG/G = \frac{EG}{\sim},\quad e\sim e'\iff \exists g\in G\text{ such that }e = e'\cdot g.$$ Here are some instances of $BG$ and $EG$. 
    \begin{center}
      \begin{tabular}{c|c|c}
        $G$ & $EG$ & $BG$\\\hline
        $\sets{e}$ & $\ast$ & $\ast$ \\ 
        $C_2$  & $S^{\infty}\subset \mR^{\oplus \infty}$ & $\mRP^\infty$ \\ 
        $\mZ$ & $\mR$ & $S^1$\\
        $S^1$ & $S^\infty \subset \mC^{\oplus \infty}$ & $\mCP^\infty$\\
        $\mathsf{SU}(2)$ & $S^\infty\subset\mH^{\oplus \infty}$ & $\mathbb{HP}^\infty$\\
        $\mathsf{O}(k)$ & $V_k(\infty)\subset (\mR^k)^{\oplus \infty}$ & $\mathsf{Gr}_k(\mR^\infty)$\\
        $\mathsf{U}(k)$ & $V_k(\infty)\subset (\mC^k)^{\oplus \infty}$ & $\mathsf{Gr}_k(\mC^\infty)$
      \end{tabular}
    \end{center}
    Some letters I wrote down are $\ortho$ the orthogonal group, $\mathsf{U}$, the unitary group and $\mathsf{SU}$ the special unitary group, $V_k$ the Stiefel manifold, $\mathsf{Gr}$ the Grassmannian, and $\mH$ the quaternions which make a 4-dimensional real vector space. 
\end{definition}

\newpage
\subsection{Homotopy groups}
\begin{fact}{\defindex{Homotopy groups of $E^{hC_2}$}}{}
  \begin{align*}
    \pi_0(E^{hC_2})&= \mZ & \pi_5(E^{hC_2})&= 0\\
    \pi_1(E^{hC_2})&= \mZ/2 & \pi_6(E^{hC_2})&=0\\
    \pi_2(E^{hC_2})&= \mZ/2 & \pi_7(E^{hC_2})&=0\\
    \pi_3(E^{hC_2})&= 0 & \pi_8(E^{hC_2})&=\mZ\\
    \pi_4(E^{hC_2})&= \mZ & \text{This }&\text{  is 8 periodic}
  \end{align*}
\end{fact}
\begin{fact}{\defindex{Homotopy groups of $E^{hC_2}\wedge V(0)$}}{}
  \begin{align*}
    \pi_0(E^{hC_2}\wedge V(0))&= \mZ/2 & \pi_5(E^{hC_2}\wedge V(0))&= 0\\
    \pi_1(E^{hC_2}\wedge V(0))&= \mZ/2 & \pi_6(E^{hC_2}\wedge V(0))&=0\\
    \pi_2(E^{hC_2}\wedge V(0))&= \mZ/4 & \pi_7(E^{hC_2}\wedge V(0))&=0\\
    \pi_3(E^{hC_2}\wedge V(0))&= \mZ/2 & \pi_8(E^{hC_2}\wedge V(0))&=\mZ/2\\
    \pi_4(E^{hC_2}\wedge V(0))&= \mZ/2 & \text{This }&\text{  is $\leq$ 8 periodic}
  \end{align*}
\end{fact}
\begin{fact}{A couple of important maps in stable homotopy}{}
  Here, when I have a finite dimensional vector space $V$, when I write $S^V$ I mean $V^+$, the one-point compactification of $V$.
  \begin{center}
    \begin{tabular}{ccc}
      $2\in \pi_0$ &$\mR^2\setminus \sets{0} \supset \mS^1\xrightarrow[2]{(x,y)\mapsto xy\inv}S^{\mR} = \mS^1$ & $C_2 = \Sigma^{-1}\Sigma^\infty \mRP^2$\\
      $\eta\in\pi_1$ & $\mC^2\setminus \sets{0} \supset \mS^3\xrightarrow[\eta]{(z_1,z_2)\mapsto z_1z_2\inv}S^{\mC} = \mS^2$ & $C_\eta = \Sigma^{-2}\Sigma^\infty \mCP^2$\\
      $\nu\in\pi_3$ & $\mH^2\setminus \sets{0} \supset \mS^7\xrightarrow[\nu]{(w_1,w_2)\mapsto w_1w_2\inv}S^{\mH} = \mS^4$ & $C_\nu = \Sigma^{-4}\Sigma^\infty \mHP^2$\\
      $\sigma\in\pi_7$ & $\mO^2\setminus \sets{0} \supset \mS^{15}\xrightarrow[\sigma]{(\omega_1,\omega_2)\mapsto \omega_1\omega_2\inv}S^{\mO} = \mS^8$ & $C_\sigma = \Sigma^{-8}\Sigma^\infty \mathbb{OP}^2$\\
    \end{tabular}.
  \end{center}
  Alternatively, note that $S^1\cong \mRP^1, S^2\cong \mCP^1, S^4 \cong \mHP^1, S^8\cong \mathbb{OP}^1$ and these maps are all $(a,b)\mapsto [a:b]$ in homogeneous coordinates. 
\end{fact}

\newpage

\newcommand{\scottycomment}[1]{\footnote{#1}}
\begin{definition}{pullbacks}{pullback}
    Let $\calC$ be a category which contains the diagram \begin{center}
      \begin{tikzcd}[ampersand replacement = \&]
        \&Y\ar[d,"g"]\\
        X\ar[r,"f"]\& Z
      \end{tikzcd}.
    \end{center}
    A \defindex{pullback} of this diagram $W$ is 3 pieces of information
    \begin{itemize}
        \item An object $\textcolor{wonglightblue}{W}\in \text{ob}(\calC)$
        \item A map $\textcolor{wonglightblue}{W\xr{p_1}}X$
        \item A map $\textcolor{wonglightblue}{W\xr{p_2}}Y$
    \end{itemize} such that 
    \begin{itemize}
        \item[-] The diagram   \begin{tikzcd}[ampersand replacement = \&]
            \textcolor{wonglightblue}{W}\ar[r,wonglightblue,"p_2"]\ar[d,wonglightblue,"p_1"']\ar[dr,phantom,"\circlearrowright"]\&Y\ar[d,"g"]\\
            X\ar[r,"f"]\& Z 
          \end{tikzcd} commutes \scottycomment{``$\circlearrowright$'' is a `long'hand for commutes, and people usually suppress it from notation.}
          \item[-] If someone hands you a commutative diagram \begin{tikzcd}[ampersand replacement = \&]
            \textcolor{wongpink}{A}\ar[r,wongpink,"h_2"]\ar[d,wongpink,"h_1"']\ar[dr,phantom,"\circlearrowright"]\&Y\ar[d,"g"]\\
            X\ar[r,"f"]\& Z 
          \end{tikzcd}, then there is a \emph{UNIQUE} map $\tilde{h}$ such that \begin{tikzcd}[ampersand replacement = \&,background color = wonglightblue!5]
            \textcolor{wongpink}{A}\ar[drr,bend left = 20,"h_2",wongpink]\ar[ddr,phantom,"\circlearrowright"]\ar[drr,phantom,"\circlearrowright"]\ar[ddr,wongpink,bend right = 20,"h_1"']\ar[dr,thick,wonggreen,"{\tilde{h}}"]\&\&\\
            \&\textcolor{wonglightblue}{W}\ar[r,wonglightblue,"p_2"]\ar[d,wonglightblue,"p_1"']\ar[dr,phantom,"\circlearrowright"]\&Y\ar[d,"g"]\\
            \&X\ar[r,"f"]\& Z 
          \end{tikzcd}.
    \end{itemize}
    \scottycomment{As a shorthand, people usually write the pullback like this: 
    \begin{center}
      \begin{tikzcd}[ampersand replacement = \&]
        W\ar[r]\ar[d]\ar[dr,phantom,"\lrcorner",very near start]\& Y\ar[d]\\
        X\ar[r]\&Z\\
      \end{tikzcd}
    \end{center}}

    \medskip
    Let \textcolor{wongorange}{$\calC$} be a category with a subcategory \textcolor{wonglightblue}{$\calD$}. We say \textcolor{wonglightblue}{$\calD$} is \defindex{closed under pullbacks} by morphisms in \textcolor{wongorange}{$\calC$} if for all arrows \textcolor{wonglightblue}{$X\xr{f}Z$} in $\textcolor{wonglightblue}{\calD}$ and for all $\textcolor{wongorange}{Y\xr{g}Z}$ in \textcolor{wongorange}{$\calC$} such that we can form the pullback   \begin{tikzcd}[ampersand replacement = \&]
        W\ar[r,"p_1",wonglightblue]\ar[d,"p_2"]\ar[dr,phantom,"\lrcorner",very near start]\& Y\ar[d,"g",]\\
        X\ar[r,"f"]\& Z
      \end{tikzcd}, then the arrow \textcolor{wonglightblue}{$W\xr{p_1}Z$} is in \textcolor{wonglightblue}{$\calD$}.
\end{definition}

\begin{definition}{pushouts}{pushout}
    Let $\calC$ be a category which contains the diagram \begin{center}
      \begin{tikzcd}[ampersand replacement = \&]
        Z\&Y\ar[from =l,"g"]\\
        X\ar[from = u,"f"]\& 
      \end{tikzcd}.
    \end{center}
    A \defindex{pushout} of this diagram $W$ is 3 pieces of information
    \begin{itemize}
        \item An object $\textcolor{wonglightblue}{W}\in \text{ob}(\calC)$
        \item A map $X\textcolor{wonglightblue}{\xr{i_1}W}$
        \item A map $Y\textcolor{wonglightblue}{\xr{i_2}W}$
    \end{itemize} such that 
    \begin{itemize}
        \item[-] The diagram   \begin{tikzcd}[ampersand replacement = \&]
            Z\&Y\ar[from =l,"g"]\ar[d,wonglightblue,"i_2"]\\
            X\ar[from = u,"f"]\ar[r,wonglightblue,"i_1"]\& \textcolor{wonglightblue}{W}
          \end{tikzcd} commutes 
          \item[-] If someone hands you a commutative diagram \begin{tikzcd}[ampersand replacement = \&]
            Z\&Y\ar[from =l,"g"]\ar[d,wongpink,"\ell_2"]\\
            X\ar[from = u,"f"]\ar[r,wongpink,"\ell_1"]\& \textcolor{wongpink}{A}
          \end{tikzcd}, then there is a \emph{UNIQUE} map $\tilde{\ell}$ such that \begin{tikzcd}[ampersand replacement = \&,background color = wonglightblue!5]
            Z\&Y\ar[from =l,"g"]\ar[d,wonglightblue,"i_2"]\ar[ddr,bend left = 20,"\ell_2",wongpink]\&\\
            X\ar[from = u,"f"]\ar[r,wonglightblue,"i_1"]\ar[drr,bend right = 20,"\ell_1"',wongpink]\& \textcolor{wonglightblue}{W}\ar[dr,wonggreen,"\tilde{\ell}",thick]\&\\
            \&\& \textcolor{wongpink}{A}
          \end{tikzcd}.
    \end{itemize}
    \scottycomment{As a shorthand, people usually write the pushout like this: 
    \begin{center}
        \begin{tikzcd}[ampersand replacement = \&]
            Z\ar[dr,"\ulcorner",very near end,phantom]\&Y\ar[from =l,"g"]\ar[d,"i_2"]\\
            X\ar[from = u,"f"]\ar[r,"i_1"]\& {W}
          \end{tikzcd}
    \end{center}}
\end{definition}